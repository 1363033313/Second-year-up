\documentclass[a4paper,11pt,UTF8]{article}
\usepackage{ctex}
\usepackage{amsmath,amsthm,amssymb,amsfonts}
\usepackage{amsmath}
\usepackage[a4paper]{geometry}
\usepackage{graphicx}
\usepackage{microtype}
\usepackage{siunitx}
\usepackage{booktabs}
\usepackage[colorlinks=false, pdfborder={0 0 0}]{hyperref}
\usepackage{cleveref}
\usepackage{esint} 
\usepackage{graphicx}
\usepackage{ragged2e}
\usepackage{pifont}
\usepackage{extarrows}
\usepackage{float}
\usepackage{caption}
\usepackage{bm}
\usepackage{multirow}
\usepackage{subfigure}
\usepackage{titlesec}
\usepackage[mathscr]{eucal}
\captionsetup[figure]{name={Figure}}
\titleformat{\section}{\Large\bfseries}{\thesection}{1em}{}
\titleformat{\subsection}{\large\bfseries}{\thesubsection}{1em}{}
\titleformat{\subsubsection}{\normalsize\bfseries}{\thesubsubsection}{1em}{}
%opening
\title{《共产党宣言》读书报告}
\author{电子信息与通信学院\\谢悦晋\quad 提高2201班\quad U202210333\\QQ: 1363033313}
\date{\today}
\begin{document}
	\maketitle

在读完《共产党宣言》之前,我问了自己一个问题:到底什么是无产者、社会主义、共产主义?真正想过这些问题后我才发现,我似乎从来没有真正了解过这些耳熟能详的词汇,他们无时无刻的出现在我的生活中,但我却无法说出这些伟大的思想究竟代表什么,一个出生生长于社会主义大地的新时代大学生,却连自己祖国的根基思想都无法解释清楚,这确实让我感慨万千。读完一本不到60页的《共产党宣言》,而我心中的收获却是一定远远大于这冷冰冰的数字的!

《共产党宣言》是由德国哲学家卡尔·马克思和弗里德里希·恩格斯共同创作的一部政治宣言。它诞生于1848年的欧洲,当时的
欧洲和北美大陆正处于从农业经济向初期工业化经济快速转变时期,社会生产力出现井喷式增长。资本主义大工业的发展和工厂制度的普遍建立,使资本主义社会的基本矛盾日益暴露出来,资产阶级与无产阶级的阶级矛盾异常尖锐。残酷的剥削、压迫与极度困窘的生存状况必然激起无产阶级的强烈反抗与斗争。因此在19世纪30~40年代,欧洲爆发了一系列的革命运动,如法国七月革命、里昂工人起义、英国宪章运动、德国和意大利的民族统一运动等,这些运动都反映了资产阶级和无产阶级的不同利益和诉求。但是由于当时无产阶级缺少纲领性文件的领导,加上缺乏经验,这些运动都受到了重创,也为马克思和恩格斯提供了丰富的实践材料和经验教训,使他们能够更深刻地认识资本主义社会的本质和无产阶级革命的方向。

马克思和恩格斯在他们的青年时期是革命民主主义者。他们在这些活动中,逐渐完成了由革命民主主义者向共产主义者的转变,他们认识到无产阶级是推翻资本主义制度的最有力的革命阶级,共产主义是无产阶级的最高理想和目标,无产阶级需要一个自己的政党来领导革命斗争。马克思和恩格斯吸收了当时的一些先进的理论成果,如英国古典政治经济学、黑格尔的辩证法、费尔巴哈的唯物主义等,同时也批判了各种假社会主义,如封建的社会主义、小资产阶级的社会主义、空想社会主义等。马克思和恩格斯在与这些思潮的斗争中,逐步形成了自己的科学社会主义理论,即历史唯物主义和剩余价值理论,为无产阶级提供了一种全新的世界观和方法论。因此,他们接受了共产主义者同盟的委托,为该组织起草了《共产党宣言》,这是他们将科学理论运用于工人革命运动、创立无产阶级政党的实践的产物。

当我第一次翻开《共产党宣言》,粗略的浏览他的文章架构组成时,我惊讶的发现,不到60页的内容竟然有七篇序言,占据了文章接近一半的内容,我起初认为这或许过于冗余,但读下来却发现这七篇序言正见证了共产主义的发展。 每一篇序言都诞生于不同的历史背景,从1872年德文序言到1893年意大利文序言,这七篇序言见证了马克思主义的出传播。《宣言》在各个国家有了发行版本并传播。这也从侧面说明了在资本主义与社会生产力快速提升同时,各国工人运动也空前高涨,十分需要一个强有力的纲领性文件作为理论指导。一版一版的序言,我能从中感受到马克思恩格斯对于共产主义必胜的坚定信念,他们充满希冀而又锋利的眼光似乎还在注视世界大地。

《共产党宣言》正文总共分为四章:资产者和无产者、无产者和共产党人、社会主义的和共产主义的文献、共产党人对各种反对党派的态度。它无情地揭露了资本主义的本质--对无产阶级进行直接的剥削,鲜明指出了资产阶级必将灭亡,无产阶级必然胜利的结论。《共产党序言》阐明了马克思的阶级斗争的学说,讲述了无产阶级政党的性质、特点、目的和任务,以及共产党的纲领和理论,抨击了当时流行的各种假社会主义,说明了各种假社会主义流派产生的社会历史条件,并揭露了它们的阶级实质,表明了共产党人革命斗争的思想策略。在工人运动士气高涨的欧洲,共产党宣言的出版仿佛一盏明灯,他指引了共产主义的发展方向,字字透露着共产主义必胜的坚定决心他鼓舞着广大无产阶级的士气,为了无产阶级的解放奋斗,他又像一口棺材,一口专门为资产阶级准备的棺材,他宣告了资本主义灭亡的必然性,用犀利的语言与资产阶级针锋相对,一字一句将资本主义送进棺材。

共产党宣言虽然文章不长,但它其中蕴含的思想却是十分丰厚的,其实在读完共产党宣言时,我发现我自己仍然不能对自己起初的问题做出一个完整的回答,这正是因为其中蕴含的思想过于充盈了,难以一次性的全盘吸收,在查阅各类资料后,我重新梳理了《》宣言文章结构,才对自己的疑惑有了一个较好自我理解。作为一个在上马原课之前完全没有系统接触过马克思主义理论的学生,通过马原课的学习和《共产党宣言》的阅读,我想我的思维境界也有了很大的提升,最基础我对马克思主义的基本理论有了一个较为清除的认知。《共产党宣言》虽然写于170多年前,但其中的蕴含的马克思主义思想仍然是社会主义的核心内容。

《共产党宣言》正确性的最好证明就是俄国十月革命和中国的新民主主义革命,他们正是因为以马克思主义为指导。才获得了如此显著成果。中国在现代的持续发展也证明了《宣言》中马克思主义的正确性。

《宣言》的内容不仅仅是一件“历史文物”,它仍然预言者后来的历史和我们今天所面对的各种情况。我感觉《共产党宣言》指出的情况并不像是174年前的事情。阅读《共产党宣言》,其实很容易将其中的文字与现实联系起来,因为自《宣言》出版以来,如今世界的很多形势仍然和《宣言》相对应。
\begin{enumerate}
	\item \textbf{统治阶级的思想就是统治思想}
	
	“法律、道德、宗教在他们看来全都是资产阶级偏见,隐藏在这些偏见后面的全都是资产阶级利益”。统治阶级将利用任何道德理念来掩盖他们所统治的残酷剥削制度。比方说,虽然欧洲社会号称“福利国家”,但实际上资本家却乐于煽动对“福利懒汉”的憎恨,侃侃而谈地只是说这些人工作不够努力。然而,正是资本家坐在办公室里赚钱,而工人们辛勤劳动每周却只得85英镑。我们去笑话那些社会中最脆弱的人,对我们来说是确实一个有效的消遣手段,但是资本家们则坐在那里享受他们的亿万财富。
	
	\item \textbf{资本主义的固有危机}
	
	“生产的不断变革,一切社会状况不停的动荡,永远的不安定和变动,这就是资产阶级时代不同于过去一切时代的地方”。最近的时期确实符合这一特征。
	
	2008年的金融危机就是一个典型的例子。它给世界经济带来了严重的影响,它导致了紧缩政策的实施和社会保障的减少,对无产阶级造成了很大的困扰。同时,巨额债务不断累积,给下一代带来了负担。学生学费不断上涨,年轻人找不到稳定的工作,失业率上升,普通家庭生活条件恶化,无家可归者增加。这些问题每天都在困扰着各个资本主义国家,资本主义社会的未来前景似乎变得越来越暗淡。疫情时代只是加剧了这一过程罢了。
	
	《宣言》曾说:“资产阶级的生产关系和交换关系,资产阶级的所有制关系,这个曾经仿佛用法术创造了如此庞大的生产资料和交换手段的现代资产阶级社会,现在像一个魔法师一样不能再支配自己用法术呼唤出来的魔鬼了。”这实际上就是资本主义社会的现实写照。资本家们正在引导着国家从危机走向危机,他们正在失去对国家的控制力。
	
	\item \textbf{世界发展的各种问题}
	
	除了由资本主义危机引发的周期性破坏之外,资本主义还不断威胁着自身的存在方式。
	
	《宣言》提到了在资本主义下“人类对自然力量的制服”。在工业发展的早期阶段,资本家就开始了“化学在工业和农业中的应用”以及“整个大陆的开垦”。这导致了很严重的环境问题,如森林砍伐,伦敦雾都,全球变暖等等,也曾有报导预言过度迫害自然会在2050年之前引发气候灾难,这是我们不想看到的。
	
	帝国主义引发的野蛮战争是另一个例子。当马克思预言资产阶级“迫使一切民族——如果它们不想灭亡的话——采用资产阶级的生产方式”时,他预言了像美国和英国这样的资产阶级大国的殖民和帝国主义政策。从臭名昭著的“争夺非洲”行动到20世纪后半叶的众多战争、外国侵略和政变,我们看到了这些帝国主义国家对世界其他地区的一场虚拟“十字军东征”。
	
\end{enumerate}
除了许多令人惊讶的预言和对资本主义的合理描述之外,《宣言》的目的并不是苍白地描述资本主义有多么糟糕,他的真正目的是呼吁世界改变。最重要的是,《宣言》解释了如何与资本主义进行斗争和废除它的基础,即阶级斗争。资本主义,像任何先前的社会制度一样,产生了它自己的掘墓人,即工人阶级。如今,工人阶级比以往任何时候都更庞大、更强大!《共产党宣言》总结的思想是革命者的指南,并对现代历史上最重大的事件产生了影响。而我们站在巨人的脚步中!

《宣言》对中国的影响是巨大的,俄国十月革命一声炮响给中国送来了马克思主义。李大钊、周恩来、毛泽东,这些伟大的领袖吸收了《宣言》中的核心思想,他们高举新民主主义革命的大旗,以马克思主义为理论武器,带领中国人民获取了新民主主义革命胜利。建国之后,领袖们凭借马克思主义完成了社会主义三大改造,使国家走上了正轨。改革开放后,我们的领袖们将马克思主义中国化,开辟了中国特色社会主义道路,让我们过上了富足的生活......无数的共产主义先辈们为了社会主义的实现而前仆后继,他们以各种方式各种行动证明了《宣言》的正确性,他们的精神鼓舞着我,我想:共产主义在中国大地必将实现!在以后的历史长河中,共产主义的火焰还会一直在东方大地上燃烧,并会燃烧世界的各个角落!最后:

\centering\Huge{\textbf{全世界无产者,联合起来!}}

\end{document}