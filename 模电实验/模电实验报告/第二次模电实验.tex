\documentclass[a4paper,11pt,UTF8]{article}
\usepackage{ctex}
\usepackage{amsmath,amsthm,amssymb,amsfonts}
\usepackage{amsmath}
\usepackage[a4paper]{geometry}
\usepackage{graphicx}
\usepackage{microtype}
\usepackage{siunitx}
\usepackage{booktabs}
\usepackage[colorlinks=false, pdfborder={0 0 0}]{hyperref}
\usepackage{cleveref}
\usepackage{esint} 
\usepackage{graphicx}
\usepackage{ragged2e}
\usepackage{pifont}
\usepackage{extarrows}
\usepackage{mathptmx}
\usepackage{float}
\usepackage{caption}
\usepackage{multirow}
\usepackage{subfigure}
\usepackage{titlesec}
\usepackage{makecell}
\usepackage{tabularx}

\numberwithin{equation}{subsection}

\titleformat{\section}{\Large\bfseries}{\thesection}{1em}{}
\titleformat{\subsection}{\large\bfseries}{\thesubsection}{1em}{}
\titleformat{\subsubsection}{\normalsize\bfseries}{\thesubsubsection}{1em}{}
\begin{document}
\begin{titlepage}
	\begin{center}
		\vspace*{1cm}
		\textbf{\LARGE 实验报告:集成运算放大器在信号运算方面的应用}
		\vspace{0.5cm}
		\Large 谢悦晋\quad 提高2201班\quad U20221033
		\vspace{1cm}
		\begin{figure}[H]
			\centering
			\subfigure{
				\includegraphics[scale=1]{hust}
			}
			\subfigure{
				\includegraphics[scale=1]{eic}
			}
			\caption*{}
		\end{figure}
		\vfill
		\vspace{0.8cm}
		华中科技大学 \\
		电子信息与通信学院 \\
		Oct 31st, 2023
	\end{center}
\end{titlepage}
\tableofcontents\newpage
\section{实验名称}
共源放大电路设计、仿真与实现
\section{实验目的}
\begin{itemize}
	\item 学习共源放大电路工作原理
	\item 掌握金属-氧化物-半导体场效应管的主要性能参数及其测试方法
	\item 掌握共源放大电路参数调整方法
	\item 掌握共源放大电路的基本原理与参数测量方法
	\item 掌握MOSFET共源极放大电路的安装与测试 技术
	\item 掌握Multisim软件的使用,实现共源放大电路的仿真实现
\end{itemize}
\section{实验元器件}
\begin{table}[h]
	\centering
	\begin{tabular}{|c|c|c|}
		\hline
		名称 & 型号/参数 & 数量\\
		\hline
		场效应管 & 2N7000 & 1\\
		\hline
		\multirow{3}{*}{电容} & 4.7μF & 1 \\
		\cline{2-3}
		 & 47μF & 1\\
		\cline{2-3}
		 & 1μF & 1\\
		\hline
		电位器 & 500kΩ & 1\\
		\hline
		\multirow{4}{*}{电阻} & 100kΩ & 2 \\
		\cline{2-3}
		 & 5.1kΩ & 1\\
		\cline{2-3}
		 & 51kΩ & 1\\
		\cline{2-3}
 		 & 1kΩ & 1\\
		\hline 		
	\end{tabular}
\end{table}
\section{实验任务}
主要为以下三个实验任务:MOSFET输出特性曲线仿真、MOSFET转移特性曲线仿真、MOSFET共源放大电路安装、调试及测试
\subsection{MOSFET输出特性曲线仿真}
使用 OrCAD/Spice 分析绘制 MOSFET (2N7000) 的共源极输出特性曲线。实验步骤与要求如下:

(1)建立新项目,绘出电路图。

首先新建一个工程项目,然后放置元器件(M2N7000、Vdc、0 (GRD)等)、连线,画出如图 3.3.5 所示的电路,并在 MOSFET 的漏极放置电流测试探针

\begin{minipage}[t]{0.6\textwidth}
	(2)设置仿真简表。
	
	\ding{172} 新建仿真简表 (New Simulation Profile), 设置直流扫描分析(DC Sweep) 的主扫描(Primary Sweep), 扫描变量为VDD, 采用线性扫描,由 OV 开始至 8V 结束,步进为  0.01V。
	
	\ding{173} 设置直流扫描分析(DC Sweep)中的二级扫描(Secondray Sweep), 扫描变量为 VGG, 采用线性扫描,由 1.7V 开始至 2.05V 结束,步进为 0.05V。
\end{minipage}
\begin{minipage}[t]{0.4\textwidth}
	\begin{figure}[H]
		\centering
		\includegraphics[width=0.75\textwidth]{2.1}
		\caption{特性曲线仿真电路}
	\end{figure}
\end{minipage}


(3)保存文档、执行仿真(Run)。运行后自动打开结果显示窗,显示输出特性曲线($i_\mathrm{D}$ $v_\mathrm{DS}$)。多根曲线对应$v_\mathrm{GS}$ 的间隔为 0.05V。

(4) 将仿真结果反映至实验报告中。

\ding{172} 选中仿真电路图,复制粘贴到实验报告文档中。

\ding{173} 在结果显示窗中,选择 Window\textbackslash Copy to Clipboard...将曲线复制到剪贴板,期间最好选择“change all colors to black”将所有曲线都变为黑色。然后粘贴至实验报告文档。
\subsection{MOSFET转移特性曲线仿真}
使用 OrCAD/Spice 分析绘制 MOSFET (2N7000) 的共源极转移特性曲线。实验步骤与要求如下:

(1)修改电路参数,将$v_\mathrm{op}$电压改为 8V。

(2)设置仿真简表。新建仿真简表(New Simulation Profile), 设置直流扫描分析(DC Sweep)
的主扫描 (Primary Sweep), 扫描变量为 V$_{GG}$, 采用线性扫描,由 OV 开始至 4V 结束,步进为0.01V。

(3)保存文档、执行仿真 (Run)。运行后自动打开结果显示窗,显示转移特性曲线 $(i_\mathrm{D}-v_{\mathrm{GS}})$.

(4) 将仿真结果复制粘贴到实验报告文档中。

\subsection{MOSFET共源放大电路安装、调试及测试}
\begin{figure}[H]
	\centering
	\includegraphics[width=0.6\textwidth]{2.2}
	\caption{共源极放大电路}
\end{figure}

实验步骤与要求如下:

(1)测试电路的静态工作点。 

\ding{172} 按照图3.3.6在面包板上组装电路,$\nu_\mathrm{DD}$的 12V 取自直流稳压电源。安装电阻前先用万用表测试电阻值,填入表 3.3.2 相应栏中。检查无误后接通电源。用数字万用表的直流电压挡测量电路的 $V_\mathrm{G}$ (栅极对地电压)、$V_\mathrm{S}$(源极对地电压)和 $V_\mathrm{D}$(漏极对地电压), 计算静态工作点$Q(I_DQ$、$V_\mathrm{GSQ}$、$V_\mathrm{DSQ}$)。将结果填入表 3.3.2 相应栏中。

\ding{173}关闭电源,将 $R_\mathrm{gl}$ 改为 100k, 检查无误后接通电源,再次测量 $V_\mathrm{G}$、$V_\mathrm{s}$ 和 $V_\mathrm{D}$, 计算静态工作点$\varrho$($I_\mathrm{bQ},V_\mathrm{GSQ},V_\mathrm{DSQ})$。将结果填入表 3.3.2 相应栏中。

\ding{174} 关闭电源,将 $R_\mathrm{gl}$ 恢复为 240k, 而将 $R_{\mathrm{g2}}$ 改为 33k, 检查无误后接通电源,测量 $V_{\mathrm{G}}$、 $V_\mathrm{s}$和$V_\mathrm{D}$, 计算静态工作点$Q$($I_\mathrm{DQ}$、$V_\mathrm{GSQ}$、$V_\mathrm{DSQ}$)。完成表 3.3.2 的内容。
\begin{table}[h]
	\centering
	\resizebox{\linewidth}{!}{\begin{tabular}{|c|c|c|c|c|c|c|c|}
		\hline
		\multirow{2}{*}{} & \multicolumn{3}{c|}{实测值} & \multicolumn{3}{c|}{计算值} & \multirow{2}{*}{\shortstack{MOSFET处于\\哪个工作区}}\\
		\cline{2-7}
		& \small$V_{G}$/V & \small$V_{S}$/V & \small$V_{D}$/V & \small{$I_{DQ}=V_S/R_S$/mA}& \small$V_{GSQ}=(V_G-V_S)$/V & \small$V_{DSQ}=(V_D-V_S)$/V & \\
		\hline
		\small\makecell[l]{$R_{g1}=240k$\\$R_{g2}=100k$} &&&&&&&\\	
		\hline
		\small\makecell[l]{$R_{g1}=100k$\\$R_{g2}=100k$} &&&&&&&\\	
		\hline
		\small\makecell[l]{$R_{g1}=240k$\\$R_{g2}=33k$} &&&&&&&\\
		\hline
		实测电阻值 &\multicolumn{7}{c|}{$R_{g1}=\qquad$,$R_{g2}=\qquad$,$R_{d}=\qquad$,$R_{s}=\qquad$}\\
		\hline				
	\end{tabular}
}
\caption{静态工作点}
\end{table}
(2)测试放大电路的输入、输出波形和通带电压增益。参考上节的图 3.2.7, 搭建放大电路实验测试平台。关闭电源,将电阻参数恢复为
$R_{g1}=240k$, $R_{g2}$=100k, 检查无误后接通电源。调整信号源,使其输出峰-峰值为 30mV、频率为1kHz 的正弦波,作为放大电路的 $v_\mathrm{i}$。分别用示波器的两个通道同时测试 $v_\mathrm{i}$ 和 $v_\mathrm{o}$, 在实验报告上定量画出$v_\mathrm{i}$和$v_\mathrm{o}$的波形(时间轴上下对齐), 分别测试负载开路和 $R_\mathrm{L}=5$.1k$\Omega$两种情况下的 $v_{7}$ 和$v_{0}$,完成表 3.3.3。

(3)测试放大电路的输入电阻。采用在输入回路串入已知电阻的方法测量输入电阻。由于 MOSFET 放大电路的输入电阻
较大,所以当测量仪器的输入电阻不够大时,采用如图 3.2.8 所示的方法可能存在较大误差, 改用如图 3.3.7 所示的测量输出电压的方法更好。$R$ 取值尽量与 $R_i$ 接近(此处可取 R=51k$\Omega$)。信号源仍旧输出峰-峰值 30mV、1kHz 正弦波,用示波器的一个通道始终监视 $v_\mathrm{i}$ 波形,用另个通道先后测量开关 S 闭合和断开时对应的输出电压 $\upsilon_\mathrm{ol}$ 和 $\upsilon_\mathrm{o2}$, 则输入电阻为
\begin{align}R_{\mathrm{i}}=\frac{v_{\mathrm{o}2}}{v_{\mathrm{ol}}-v_{\mathrm{o}2}}\cdot R\end{align}

测量过程要保证 $v_{\mathrm{o}}$不出现失真现象
\begin{figure}[H]
	\begin{minipage}[h]{0.6\textwidth}
		\centering
		\captionsetup{labelsep=space, textformat=simple, format=plain} 
		\renewcommand{\figurename}{表}  
		\resizebox{\linewidth}{!}{\begin{tabular}{|c|c|c|c|c|c|}
				\hline
				\shortstack{负载\\情况} & \shortstack{$v_i$峰-峰\\值$V_{ipp}$/mV} & \shortstack{$v_o$峰-峰值\\$V_{opp}$/mV} & \shortstack{$|A_v|=$\\$V_{opp}/V_{ipp}$}	& \shortstack{$|A_v|$的\\理论值} & \shortstack{相对\\误差}\\
				\hline
				负载开路 & 30 &&&&\\
				\hline
				$R_L=5.1\mathrm{k\Omega}$ & 30 &&&&\\
				\hline		
			\end{tabular}
		}
		\caption{:电压增益($f\mathrm{=}\mathrm{1kHz}$)}
	\end{minipage}
	\begin{minipage}[h]{0.4\textwidth}
		\centering
		\includegraphics[width=\textwidth]{2.3}
		\caption{高输入电阻测试局部示意图}
	\end{minipage}
\end{figure}

(4)测试放大电路的输出电阻。

采用改委负载的方法测试输出电阻。分别测试负载开路输出电压 $v_o^{\prime}$ 和接入已知负载$R_L$时的输出电压 $v_\mathrm{o}$,测量过程同样要保证 $v_\mathrm{o}$ 不出现失真现象。实际上在表 3.3.3 中已得到 $v_\mathrm{o}^{\prime}$ 和$v_0$, 则输出电阻为
\begin{align}
	R_\mathrm{o}=\frac{v_\mathrm{o}^{\prime}-v_\mathrm{o}}{v_\mathrm{o}^{\prime}}\times R_\mathrm{L}
\end{align}

$R_{\mathrm{L}}$越接近 $R_{0}$ 误差越小。

(5)测试放大电路的通频带。在图3.3.6中,输入$v_\mathrm{i}$ 为峰-峰值 30mV、1kHz 的正弦波,用示波器的一个通道始终监视输入波形的峰-峰值,用另一个通道测出输出波形的峰-峰值。保持输入波形峰-峰值不变,调节信号源的频率,逐渐提高信号的频率,观测输出波形的幅值变化,并相应适时调节示波器水平轴的扫描速率,保证始终能清晰观测到正常的正弦波。持续提高信号频率,直到输出波形峰峰值降为 1kHz 时的 0.707 倍,此时信号的频率即为上限频率 $f_\mathrm{H}$, 记录该频率; 类似地,逐渐降低信号频率,直到输出波形峰-峰值降为 1kHz 时的 0.707 倍,此时的频率即为下限频率 $f_\mathrm{L}$, 记录该频率,完成表 3.3.4。要特别注意,测试过程必须时刻监视输入波形峰-峰值,若有变化,需调整信号源的输出幅值,保持$v_\mathrm{i}$的峰-峰值始终为 30mV。

通频带(带宽)为: 
\begin{align}\mathrm{BW}=f_{\mathrm{H}}-f_{\mathrm{L}}\end{align}
\begin{table}[H]
	\centering
	\begin{tabular}{|c|c|c|c|}
		\hline
		\multirow{2}{*}{信号频率$f$} & $f_L$ & - & $f_H$\\
		\cline{2-4}
		&&1kHz&\\	
		\hline
		\shortstack{输出波形\\峰-峰值$V_{opp}$}&&&\\
		\hline
		$|A_v|$&&&\\
		\hline	
	\end{tabular}
	\caption{通频带($V_{\mathrm{ipp}}=30$mV)}
\end{table}
\section{实验原理}
\subsection{MOSFET共源放大电路安装、调试及测试}
\begin{minipage}[t]{0.6\textwidth}
	图 3.3.6 为 N 沟道增强型 MOSFET 共源极放大电路,其静态工作点可由式(4.3.1) 估算
	\begin{subequations}\begin{align}
			V_{\mathrm{GSQ}}=\frac{R_{\mathrm{g2}}}{R_{\mathrm{g1}}+R_{\mathrm{g2}}}\times V_{\mathrm{DD}}-I_{\mathrm{DQ}}R_{\mathrm{s}}  \\
			I_{\mathrm{DQ}}=K_{\mathrm{n}}\left(V_{\mathrm{GS}}-V_{\mathrm{TN}}\right)^{2} \\
			V_{\mathrm{DSQ}}=V_{\mathrm{DD}}-I_{\mathrm{DQ}}(R_{\mathrm{d}}+R_{\mathrm{s}}) 
	\end{align}\end{subequations}
	动态性能指标可由式(4.1) 估算
	\begin{subequations}\begin{align}
			A_\mathrm{v}=-g_\mathrm{m}R_\mathrm{d}\\
			R_{\mathrm{i}}=R_{\mathrm{g1}}//R_{\mathrm{g2}}\\
			R_{\mathrm{o}}=R_{\mathrm{d}}
	\end{align}\end{subequations}
\end{minipage}
\begin{minipage}[t]{0.4\textwidth}
	\begin{figure}[H]
		\centering
		\includegraphics[width=\textwidth]{2.2}
		\caption{共源极放大电路}
	\end{figure}
\end{minipage}

数据手册通常会给出$\nu_\mathrm{TN}$和某工作点下的$g_\mathrm{m}$。由表 3.3.1 看出,对于 MOS 管 2N7000,$I_{\mathrm{D} }= 200$mA 时,$g_m^{\prime}= 100$mS, 可得 $K_n= ( g_m^{\prime}/2) ^2/I_{\mathrm{D} }= 12.5$mA$/V^2$ 式(3.3.4a)中的$g_\mathrm{m}$是图 3.3.6 电路静态工作点下 MOS 管的互导,同样可得
\begin{align}
	g_m&=g_m'\sqrt{I_{\mathrm{DQ}}/I_{\mathrm{D}}}\\
	g_{\mathrm{m}}&=10\sqrt{I_{\mathrm{DQ}}/2}\mathrm{mS}
\end{align}

由数据表可知$V_{\mathrm{TN}}$在0.8-3V之间,这里取 $V_{\mathrm{TN}}=1.75\mathrm{V}$
\subsection{Multisim的使用和学习}
\section{实验过程}
\subsection{Multisim 仿真}
\subsubsection{DC Operating Point 模拟直流静态工作点}
\subsubsection{Single frequency ac analysis 得到输入输出电压曲线}
\subsubsection{AC Analysis 得到幅频特性曲线}
\subsubsection{AC 模式测量输入阻抗}
\subsubsection{AC 模式测量输出阻抗.}
\subsubsection{MOSFET 输出特性仿真}
\subsubsection{MOSFET 转移特性仿真}
\subsection{单极 MOSFET 共源放大电路插板实验}
\subsubsection{测试静态工作点}
实验中数据记录表格如下:
\begin{table}[H]
	\centering
	\resizebox{\linewidth}{!}{\begin{tabular}{|c|c|c|c|c|c|c|c|}
			\hline
			\multirow{2}{*}{} & \multicolumn{3}{c|}{实测值} & \multicolumn{3}{c|}{计算值} & \multirow{2}{*}{\shortstack{MOSFET处于\\哪个工作区}}\\
			\cline{2-7}
			& $V_{G}$/V & $V_{S}$/V & $V_{D}$/V & {$I_{DQ}=V_S/R_S$/mA}& $V_{GSQ}=(V_G-V_S)$/V & $V_{DSQ}=(V_D-V_S)$/V & \\
			\hline
			\makecell[l]{$R_{g1}=240k$\\$R_{g2}=100k$} & 3.45174 & 1.85189 & 2.59015 &  &&&\\	
			\hline
			\makecell[l]{$R_{g1}=100k$\\$R_{g2}=100k$} & 5.99462 & 1.97318 & 1.98032 &&&&\\	
			\hline
			\makecell[l]{$R_{g1}=240k$\\$R_{g2}=33k$} & 1.44178 & 0.163566 & 11.4622 &&&&\\
			\hline
			实测电阻值 &\multicolumn{7}{c|}{$R_{g1}=241.87\mathrm{k\Omega},97.839\mathrm{k\Omega}\qquad$,$R_{g2}=98.232\mathrm{k\Omega},33.132\mathrm{k\Omega}\qquad$,$R_{d}=5.0156\Omega\qquad$,$R_{s}=983.44\Omega\qquad$}\\
			\hline				
		\end{tabular}
	}
	\caption{静态工作点}
\end{table}
\subsubsection{测试放大电路的输入、输出波形和通带电压增益}
输入、输出波形如下:
\begin{figure}[H]
	\subfigure{
		\centering
		\includegraphics[width=0.49\textwidth]{TEK00004.PNG}
	}
	\subfigure{
		\centering
		\includegraphics[width=0.49\textwidth]{TEK00005.PNG}
	}
\end{figure}
实验数据记录表格如下:
\begin{table}[H]
	\centering
	\begin{tabular}{|c|c|c|c|c|c|}
			\hline
			\shortstack{负载\\情况} & \shortstack{$v_i$峰-峰\\值$V_{ipp}$/mV} & \shortstack{$v_o$峰-峰值\\$V_{opp}$/mV} & \shortstack{$|A_v|=$\\$V_{opp}/V_{ipp}$}	& \shortstack{$|A_v|$的\\理论值} & \shortstack{相对\\误差}\\
			\hline
			负载开路 & 30(39.20) & 1304&&&\\
			\hline
			$R_L=5.1\mathrm{k\Omega}$ & 30(38.40) & 222.0 &&&\\
			\hline		
		\end{tabular}
	\caption{:电压增益($f\mathrm{=}\mathrm{1kHz}$)}
\end{table}
\subsubsection{测试放大电路的输入电阻}
测量得到此时的$R=50.914\mathrm{k\Omega}$,接入电路后,输出波形如下:
\begin{figure}
	\centering
	\includegraphics[width=0.7\textwidth]{TEK00009.PNG}
\end{figure}

根据公式(4.3.1)可以计算输入电阻为:$R_i=$
\subsubsection{测试放大电路的输出电阻}
根据表4以及公式(4.3.2)可以计算输出电阻为:$R_o=\mathrm{k\Omega}$
\subsubsection{测试放大电路的通频带}
下表记录了通频带的测量数据:
\begin{table}[H]
	\centering
	\resizebox{\linewidth}{!}{\begin{tabular}{|c|c|c|c|c|c|c|c|c|c|c|c|c|c|}
		\hline
		$f$/Hz&5&10&15&20&30&40&50&70&100&200&500&2k&30k\\	
		\hline
		$V_{opp}$/mV&420.0&260.0&600.0&700.0&920.0&1040&1120&1180&1240&1280&1288&1288&1288\\
		\hline
		100k&200k&250k&260k&270k&275k&280k&285k&290k&300k&310k&330k&350k&370k\\
		\hline
		1240&1080&960.0&940.0&920.0&920.0&912.0&900.0&880.0&860.0&840.0&820.0&780.0&760.0\\
		\hline
		400k&450k&500k&600k&800k&1M&&&&&&&&\\
		\hline
		700.0&680.0&640.0&560.0&440.0&360.0&&&&&&&&\\
		\hline
	\end{tabular}}
	\caption{通频带($V_{\mathrm{ipp}}=30$mV)}
\end{table}
\section{实验小结}
\end{document}