\documentclass[a4paper,11pt,UTF8]{article}
\usepackage{ctex}
\usepackage{amsmath,amsthm,amssymb,amsfonts}
\usepackage{amsmath}
\usepackage[a4paper]{geometry}
\usepackage{graphicx}
\usepackage{microtype}
\usepackage{siunitx}
\usepackage{booktabs}
\usepackage[colorlinks=false, pdfborder={0 0 0}]{hyperref}
\usepackage{cleveref}
\usepackage{esint} 
\usepackage{graphicx}
\usepackage{ragged2e}
\usepackage{pifont}
\usepackage{extarrows}
\usepackage{float}
\usepackage{caption}
\usepackage{bm}
\usepackage{multirow}
\usepackage{subfigure}
\usepackage{titlesec}
\usepackage[mathscr]{eucal}


\captionsetup[figure]{name={Figure}}
\titleformat{\section}{\Large\bfseries}{Chapter \thesection}{1em}{}
\titleformat{\subsection}{\large\bfseries}{\thesubsection}{1em}{}
\titleformat{\subsubsection}{\normalsize\bfseries}{\thesubsubsection}{1em}{}
%opening
\title{2023年科学计算引论(计算方法)试题回忆版本}
\date{\today}
\begin{document}
\maketitle

\noindent 一、填空题(共20分,每题4分)
\begin{enumerate}
	\item $\dfrac{22}7,\dfrac{355}{113}$分别是$\pi$的近似值,有几位有效数字 
	
	\item $e^{-x^3}+2\sin (\dfrac12\pi x)-1.5=0$在[0,1]上若需要满足$\varepsilon<1\times 10^{-5}$,则至少需迭代多少次,$[-1,0]$上是否能用二分法求根?
	
	\item $f(x)=2x^3-3x^2+1,f[0,1]=,f[-1,0,1,2]=$
	
	\item 给出求$\sqrt[n]{a}$的Newton迭代公式,用其计算$\sqrt[3]{4}$的近似值$\varepsilon<0.001$
	
	\item 给出下列求积公式的代数精度$$
	\int_{-1}^1f(x)\mathrm{d}x=\frac12f(-0.5)+f(0)+\frac12f(0.5)$$
	$$
	\int_{-1}^1f(x)\mathrm{d}x=\frac13f(-1)+\frac43f(0)+\frac12f(1)
	$$
\end{enumerate}
\noindent 二、给定函数$f(x)=2x^3-2x^2-x-1=0$

(1)证明$f(x)$在$[0,1]$上有零点

(2)确定下列迭代各式在$x_0=0.5$处是否收敛

(a)
$$
x_{k+1}=2x_k^3-2x_k^2-1
$$

(b)
$$
x_{k+1}=\sqrt{\frac12(2x_k^3-x_k-1)}
$$

(3)用Newton迭代求其根,初值选取$x_0=0.5,\varepsilon<0.001$

\noindent 三、给定下列插值节点及函数值:
\begin{table}[h]
	\centering
	\begin{tabular}{|c|c|c|c|c|}
		\hline
		$x_i$ &-1  &  0   &  1   &  3   \\
		\hline
		$y_i$ & -30  & -12  &  -8  &  30  \\
		\hline
	\end{tabular}
\end{table}

(1)计算Lagrange插值多项式$L_3(x)$

(2)计算Newton插值多项式$N_3(x)$

(3)计算$L_3(2),N_3(2),L_3(0.5),N_3(0.5)$

\noindent 四、给定求积公式
$$
\int_{-1}^1f(x)\mathrm{d}x\approx A_0f(x_0)+A_1f(x_1)
$$

(1)确定$A_0,x_0,A_1,x_1$, 使其有最高的代数精度,并给出求积公式的代数精度

(2)用上述求积公式计算下列积分
$$
\int_{0}^1x^2e^x\mathrm{d}x
$$

\noindent 五、用Romberg求积公式计算下列积分,$\varepsilon<0.001$

$$
\int_0^1\frac{x^2}{1+x}\mathrm{d}x
$$

\noindent 六、用Gauss-Seidel迭代法计算下列方程组,并使两次迭代之间的最大误差$\varepsilon<0.001$
$$
\begin{pmatrix}
	10&-1&0&1\\
	0&10&1&0\\
	1&2&20&1\\
	0&2&-1&20\\
\end{pmatrix}\begin{pmatrix}
x_1\\
x_2\\
x_3\\
x_4\\
\end{pmatrix}=\begin{pmatrix}
	-11\\
	20\\
	4\\
	24\\
\end{pmatrix}
$$

\noindent 七、给定初值问题如下
$$
\begin{cases}
	u^\prime=-u+\dfrac{t}2,,t\in[0,1]\\
	u(0)=1
\end{cases}
$$

(1)利用显式Euler格式以及中点矩形格式,构造一个显式预估-校正格式

(2)用上述显式预估-校正格式解上述初值问题


\end{document}    