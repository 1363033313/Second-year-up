\documentclass[a4paper,11pt,UTF8]{article}
\usepackage{ctex}
\usepackage{amsmath,amsthm,amssymb,amsfonts}
\usepackage{amsmath}
\usepackage[a4paper]{geometry}
\usepackage{graphicx}
\usepackage{microtype}
\usepackage{siunitx}
\usepackage{booktabs}
\usepackage[colorlinks=false, pdfborder={0 0 0}]{hyperref}
\usepackage{cleveref}
\usepackage{esint} 
\usepackage{graphicx}
\usepackage{ragged2e}
\usepackage{pifont}
\usepackage{extarrows}
\usepackage{mathptmx}
\usepackage{float}
\usepackage{caption}
\usepackage{bm}
\usepackage{multirow}
\usepackage{subfigure}
\usepackage{titlesec}
\captionsetup[figure]{name={Figure}}
%opening
\title{科学计算引论作业(五)}
\author{谢悦晋 \quad U202210333}
\date{Oct 22nd, 2023 }
\begin{document}
\maketitle
\textbf{4.8} 给定 $f(x)=\sinh x$ 及插值节点 $x_0=0.40,x_1=0.55,x_2=0.70,x_3=0.85,x_4=1.00$, 试构造 4次 Newton 插值多项式计算 $f(0.596)$ 的通近值,并指出其绝对误差。

解:

注意到插值点实际上是等间距的,故可以应用等距节点时的Newton插值公式,差分表如下:
\begin{table*}[h]
	\centering
	\caption{4.2 差分表}
	
	\begin{tabular}{|c|c|c|c|c|c|}
		\hline
		$x$ & $f(x)$ & $\Delta$ & $\Delta^2$ & $\Delta^3$ & $\Delta^4$\\
		\hline
		$0.40$ & 0.410752 &  &&&\\
		\hline
		$0.55$ & 0.578151 & 0.167399 &&&\\
		\hline
		$0.70$ & 0.758583 & 0.180432 & 0.013033&&\\
		\hline
		$0.85$ & 0.956115 & 0.197532 & 0.017100 & 0.004067&\\
		\hline
		$1.00$ & 1.175201 & 0.219085 & 0.021553 & 0.004453& 0.000385\\
		\hline
	\end{tabular}
\end{table*}
由差商表易得Newton插值多项式:
$$\begin{aligned}
	N_4(x)=N_4(x_0+0.15t)&=0.410752+0.167399t+\frac{0.013033}{2!}t(t-1)+\frac{0.004067}{3!}t(t-1)(t-2)\\&+\frac{0.000385}{4!}t(t-1)(t-2)(t-3)
\end{aligned}
$$
计算结果如下:
$$
	N_4(0.596)=N_4(0.4+0.15\frac{98}{75})=0.6319171542,\\ |f(0.596)-N_4(0.596)|=3.523342521072337\times10^{-7}
$$
\end{document}