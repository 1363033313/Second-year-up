\documentclass[a4paper,11pt,UTF8]{article}
\usepackage{ctex}
\usepackage{amsmath,amsthm,amssymb,amsfonts}
\usepackage{amsmath}
\usepackage[a4paper]{geometry}
\usepackage{graphicx}
\usepackage{microtype}
\usepackage{siunitx}
\usepackage{booktabs}
\usepackage[colorlinks=false, pdfborder={0 0 0}]{hyperref}
\usepackage{cleveref}
\usepackage{esint} 
\usepackage{graphicx}
\usepackage{ragged2e}
\usepackage{pifont}
\usepackage{extarrows}
\usepackage{mathptmx}
\usepackage{float}
\usepackage{caption}
\usepackage{bm}
\usepackage{multirow}
\usepackage{subfigure}
\usepackage{titlesec}
\captionsetup[figure]{name={Figure}}
%opening
\title{科学计算引论作业(七)}
\author{谢悦晋 \quad U202210333}
\date{Nov 6th, 2023 }
\begin{document}
\maketitle
\textbf{5.1} 试分别利用中矩公式、梯形公式及 Simpson 公式计算定积分
$$
I=\int_0^{\frac12}\exp(3x)\cos2x\mathrm{d}x,
$$
并比较其计算精度。

\textbf{5.2} 试确定下面求积公式
$$
\int_{-1}^1f(x)\mathrm{d}x\approx a[f(x_0)+f(x_1)+f(x_2)],
$$

使其具有 3 次代数精度,并由该公式计算定积分:
$$
	\int_{-1}^1\frac{x\sin x}{\sqrt{1+x^2}}\mathrm{d}x
$$

\textbf{5.5} 试构造两点 Gauss 型求积公式
$$
	\int_{-1}^1f(x)\mathrm{d}x\approx A_0f(x_0)+A_1f(x_1),
$$

并由此计算积分:
$$
	\int_0^1\sqrt{1+2x}\mathrm{d}x
$$

\textbf{5.7} (实验题) 分别用变步长梯形求积公式和 Romberg 算法计算椭圆积分
$$
	\int_0^\pi\frac{\sqrt{2}}{(1+\sin^2x)\sqrt{2-\sin^2x}}\mathrm{d}x
$$

要求其逼近值 $T_k,T_k^{(0)}$ 的计算精度分别满足 $|T_k-T_{k-1}|<10^{-12}$ 和$|T_k^{(0)}-T_{k-1}^{(0)}|<10^{-12}$.
\end{document}