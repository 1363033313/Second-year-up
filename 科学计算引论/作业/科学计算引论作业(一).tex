\documentclass[a4paper,11pt,UTF8]{article}
\usepackage{ctex}
\usepackage{amsmath,amsthm,amssymb,amsfonts}
\usepackage{amsmath}
\usepackage[a4paper]{geometry}
\usepackage{graphicx}
\usepackage{microtype}
\usepackage{siunitx}
\usepackage{booktabs}
\usepackage[colorlinks=false, pdfborder={0 0 0}]{hyperref}
\usepackage{cleveref}
\usepackage{esint} 
\usepackage{graphicx}
\usepackage{ragged2e}
\usepackage{pifont}
\usepackage{extarrows}
\usepackage{mathptmx}
\usepackage{float}
\usepackage{caption}
\captionsetup[figure]{name={Figure}}
%opening
\title{科学计算引论作业(一)}
\author{谢悦晋 \quad U202210333}
\date{Sept 15th, 2023 }
\begin{document}
\maketitle
\noindent\textbf{1.5} 若以$\displaystyle\frac{355}{113}$作为圆周率$\pi$的逼近值,问此逼近值具有多少位有效数字\\
解:$x=\frac{325}{113}=0.314159204\ldots\times10^1$\\
$|x-\pi|=2.6676\ldots\times10^{-7}<0.5\times10^{-6}=0.5\times10^{1-7}$\\
$\therefore$有7位有效数字\\
\noindent\textbf{2.1} 使用二分法求方程$x=2^{-x}$在$[0,1]$内的根,精确到$10^{-8}$\\
解:迭代次数$\displaystyle k>\frac{\ln(b-a)-\ln2\varepsilon}{\ln2}=25.57\Rightarrow k=26$,用计算机模拟计算:
\begin{figure}[H] 
	\centering 
	\includegraphics[scale=0.423]{kx2.1.png}
	\caption{代码运行图}
\end{figure}
\noindent 迭代27,更新26次,最终结果为:$0.6411857418715954$
\end{document}