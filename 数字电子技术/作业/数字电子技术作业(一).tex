\documentclass[a4paper,11pt,UTF8]{article}
\usepackage{ctex}
\usepackage{amsmath,amsthm,amssymb,amsfonts}
\usepackage{amsmath}
\usepackage[a4paper]{geometry}
\usepackage{graphicx}
\usepackage{microtype}
\usepackage{siunitx}
\usepackage{booktabs}
\usepackage[colorlinks=false, pdfborder={0 0 0}]{hyperref}
\usepackage{cleveref}
\usepackage{esint} 
\usepackage{graphicx}
\usepackage{ragged2e}
\usepackage{pifont}
\usepackage{extarrows}
\usepackage{mathptmx}
\usepackage{float}
\usepackage{caption}
\captionsetup[figure]{name={Figure}}
%opening
\title{数字电子技术作业(一)}
\author{谢悦晋 \quad U202210333}
\date{Sept 24th, 2023 }
\begin{document}
\maketitle
\noindent\textbf{2.1.3} 应用反演规则和对偶规则,求下列函数的非函数和对偶函数:\\
(1)$L=A\cdot B+\overline{A}\cdot\overline{B}$\\
(2)$L=AB+\overline{C+D}$\\
(3)$L=\overline{A}\cdot\overline{B}+\overline{\overline{A}\cdot B\cdot\overline{C}}\cdot D$\\
\textbf{2.2.3} 试写出下列各个函数的最小项表达式:\\
(3)$L=\overline{\overline{AB}+ABD}(B+\overline{C}D)$\\
(4)$L=\overline{(A\overline{B}+B\overline{C})\overline{AB}}$\\
\textbf{2.3.1}
用代数法将下列各式化简成最简的与-或表达式\\
(1)$\overline{AB+\overline{A}\cdot\overline{B}+\overline{A}B+A\overline{B}}$\\
(2)$\overline{\overline{(\overline{A}+B)}+\overline{(A+B)}+(\overline{\overline{A}B})(\overline{A\overline{B}})}\\$
(3)$\overline{B}+ABC+\overline{AC}+\overline{AB}$\\
(4)$\overline{ABC}+A\overline{B}C+ABC+A+B\overline{C}\\$
(5)${ABC}{\overline{D}+ABD+BC}{\overline{D}+ABCD+B}{\overline{C}}$\\
(6)$\overline{{\overline{AC+\overline{A}BC}+\overline{B}C+AB\overline{C}}}$\\
\textbf{2.4.3} 用卡诺图法化简下列各式:\\
(1) $A\overline{B}CD+AB\overline{C}D+A\overline{B}+A\overline{D}+A\overline{B}C$\\
(2)$\overline{A}\cdot\overline{B}C+A\overline{B}\cdot\overline{C}D+AB\overline{C}D+ABC$\\
(3)$
A\overline{B}CD+D(\overline{B}\cdot\overline{C}D)+(A+C)B\overline{D}+\overline{A}\overline{(\overline{B}+C)}
$\\
(4)$
L(A,B,C,D)=\sum m(0,2,4,8,10,12)
$\\
(5)$
L(A,B,C,D)=\sum m(0,1,2,5,6,8,9,10,13,14)
$\\
(6)$
L(A,B,C,D)=\sum m(\begin{matrix}{0,2,4,6,9,13}\\\end{matrix})+\sum d(\begin{matrix}{1,3,5,7,11,15}\\\end{matrix})
$\\
(7)$
L(A,B,C,D)=\sum m(0,4,6,13,14,15)+\sum d(1,2,3,5,7,9,10,11
$\\ 
\textbf{2.4.4} 用卡诺图化简法,求下列函数的最简或-与表达式\\
(1)$
L(A,B,C,D)=A\overline{C}+AD+\overline{B}\cdot\overline{C}+\overline{B}D
$\\
(2)$
L(A,B,C,D)=\sum m(3,4,5,7,13,14,15)
$\\








\end{document}